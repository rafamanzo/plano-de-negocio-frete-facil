\chapter{Avaliação Estratégica}

\section{Matriz SWOT (FOFA)}
  \begin{tabular}{|p{8.0cm}|p{8.0cm}|}
  \hline
   \textbf{Forças} & \textbf{Fraquezas}\\ \hline
    Competência Técnica & Pouco Know-How do Meio \\ \hline
   Facilidade de Implementação & Capital Inicial Restrito \\ \hline
   Investimento Inicial Baixo & Adesão das Transportadoras\\ \hline
   Sem necessidade de lidar com transações financeiras diretamente &  Falta de designers e marketeiros dentre os integrantes \\ \hline
    \textbf{Oportunidades} & \textbf{Ameaças}\\ \hline
Sem serviços semelhantes no mercado & Possível Competição com grandes transportadoras\\ \hline
Transporte rodoviário é a pricipal  forma de transportar cargas no país & Garantir a confiabilidade do Sistema \\ \hline
   Preço final acessível para o Cliente & Pouca divulgação do Sistema\\ \hline
  \end{tabular}
 \section{Análise da Matriz}
 
\subsection{Forças e Oportunidades}
\begin{itemize}
 \item Aliar a competência técnica e a facilidade de implementação para aproveitar a falta de
concorrência no mercado lançando o produto rapidamente.
 \item Aproveitar o investimento inicial baixo para atingir um valor de mercado acessível para os clientes.
\end{itemize}
\subsection{Forças e Ameaças:}
\begin{itemize}
\item Garantir a confiabilidade do sistema evitando lidar diretamente com transações financeiras reduzindo o risco de fraudes e consequente perda de confiança no produto.
\end{itemize}
\subsection{Fraquezas e Oportunidades}
\begin{itemize}
\item Através de um preço acessível conquistar rapidamente  uma base de usuários afim de obter lucro para compensar o capital inicial restrito minimizando um possível déficit orçamentário.
\end{itemize}
\subsection{Fraquezas e Ameaças}
\begin{itemize}
\item Entrar em contato com as transportadoras para apresentar o sistema com o intuito de trazê-
los como clientes e evitar uma possível competição.
\item Divulgar o serviço online para reduzir os custos com publicidade devido ao capital inicial restrito.
\end{itemize}
