\chapter{Análise de Mercado}\label{mercado}

\section{Perfil dos Clientes}

  Serviço focado em micro(receita bruta de até 2.4 milhões/ano) e pequenas empresas transportadoras (receita bruta de até 16 milhões/ano) além de profissionais autônomos e cooperativas de transporte.

\section{Mercado Potencial}

  De acordo com os dados do IBGE referentes à 2005 o perfil do Transporte Rodoviário de Carga sugere que existam na área 62.789 empresas sendo que dessas 4679 possuem 20 ou mais empregados.

  Ainda segundo o Cadastro Central de Empresas do IBGE (Cempre) para o ano de 2005, é de 92\% a participação de microempresas na área de transporte de cargas ( empresas com até nove empregados, segundo um dos critérios de classificação adotados pelo Serviço Brasileiro de Apoio às Micro e Pequenas Empresas (Sebrae)), enquanto as empresas de pequeno porte (10 a 49 empregados) respondem por outros 7\%.

  Além das microempresas transportadoras o  Registro Nacional de Transportadores Rodoviários de Carga (RNTRC), mantido pela ANTT em 2008 afirma que existem cerca de 763 mil profissionais autônomos.
  
  As cooperativas de caminhoneiros autônomos também representam uma parcela dos clientes em potencial sendo que até janeiro de 2008 constava no registro da ANTT cerca de 680 cooperativas de carga.

  A receita operacional bruta do transporte rodoviário de cargas foi de R\$ 46,2 bilhões em 2005 e o valor adicionado bruto à produção de serviços da economia no período totalizou R\$ 16,3 bilhões, esses números são referentes apenas às empresas formalizadas.

\section{Concorrência}

Nesse campo de atuação não foi encontrado nenhum serviço semelhante.

\section{Definição de Arquipélago,Ilha e Nicho}
\subsection{Arquipélago}
  Fazer o matchmaking entre um cliente que deseja envir uma carga e uma transportadora com espaço vazio em alguma entrega para o mesmo destino.

\subsection{Ilha}
Transporte de cargas por meios terrestres.

\subsection{Nicho}
Cooperativas,micro e pequenas empresas transportadoras de cargas.


\section{Fornecedores}
\subsection{Amazon Web Services}
	 Conjunto de serviços oferecidos pela amazon que provêm a infraestrutura necessária para hospedar e administrar um sistema em cloud  tais como armazenamento, rotinas de backup e ferramentas de segurança reduzindo substancialmente os custos com infraestrutura e manutenção.\newline
	O custo do serviço  da AWS depende do consumo  de dados e armazenamento, ou seja, quanto mais pessoas utliizarem o sistema o custo sobe proporcionalmente não possuindo uma mensalidade ou taxa mínima.


\section{Fontes}

\href{http://www.slideshare.net/rafaelrezoliveira/micro-epequenasempresas}{Apresentação sobre Micro e Pequenas Empresas no Brasil}
\newline
\href{http://exame.abril.com.br/pme/noticias/o-raio-x-das-pequenas-empresas-brasileiras}{Matéria na Revista Exame destacando características de pequenas empresas}
\newline
\href{http://www.bndes.gov.br/SiteBNDES/export/sites/default/bndes_pt/Galerias/Arquivos/conhecimento/revista/rev2902.pdf}{Artigo da Revista do BNDES sobre o Transporte Rodoviário de Carga}
