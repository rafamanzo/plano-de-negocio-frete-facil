\chapter{Plano Financeiro}\label{financeiro}

\section{Estimativa dos investimentos fixos}

  \subsection{Máquinas e equipamentos}
  \begin{tabular}{| l | l | l | l |}
    \hline
    \textbf{Descrição} & \textbf{Quantidade} & \textbf{Valor Unitário (R\$)} & \textbf{Total (R\$)}\\ \hline
    Computador pessoal & 5 & 1500,00 & 7500,00\\ \hline
    \multicolumn{3}{| l |}{\textbf{Subtotal}} & 7500,00\\ \hline
  \end{tabular}
  
  \subsection{Móveis e utensílios}
  \begin{tabular}{| l | l | l | l |}
    \hline
    \textbf{Descrição} & \textbf{Quantidade} & \textbf{Valor Unitário (R\$)} & \textbf{Total (R\$)}\\ \hline
    Cadeira & 5 & 100,00 & 500,00\\ \hline
    Mesa & 5 & 150,00 & 750,00\\ \hline
    \multicolumn{3}{| l |}{\textbf{Subtotal}} & 1250,00\\ \hline
  \end{tabular}

  \subsection{Total}
  Então, totalizando um investimento inicial de R\$8750,00
  
\section{Capital de giro}

  \subsection{Infraestrutura compuitacional inicial}
  
  \begin{tabular}{| l | l | l | l |}
    \hline
    \textbf{Descrição} & \textbf{Quantidade} & \textbf{Valor Unitário (R\$)} & \textbf{Total (R\$)}\\ \hline
    Reserva de 1 ano - EC2 small & 3 & 334,00 & 1002,00\\ \hline
    Reserva de 1 ano - RDS small & 1 & 140,00 & 140,00\\ \hline
    \multicolumn{3}{| l |}{\textbf{Total}} & 1142,00\\ \hline
  \end{tabular}
  
  \subsection{Estimativas de entrada e saída}
  
    \subsubsection{Prazo médio de vendas}
    
    \begin{tabular}{| l | l | l | l |}
      \hline
      \textbf{Fonte} & \textbf{\% da receita} & \textbf{Dias para receber} & \textbf{Média ponderada de dias}\\ \hline
      Assinatura & 75 & 0 & 0\\ \hline
      Participação no frete & 25 & 30 & 7,5\\ \hline
      \multicolumn{3}{| l |}{\textbf{Prazo médio total (dias)}} & 7.5\\ \hline
    \end{tabular}
    
    \subsubsection{Prazo médio de compras}
    Como as compras basicamente serão de infraestrutura na nuvem, cuja forma de pagamento, em geral, é por cartão de crédito internacional, podemos considerar um prazo médio de pagamento de 15 dias até o fechamento da fatura.
    
  \subsection{Totalizando}
  Portanto, a empresa deve receber os recursos para pagar suas contas antes de, de fato, ter que pagá-las, assim sendo a empresa, desde que tudo corra bem, não tem necessidade um caixa mínimo para realizar suas atividades.
  
\section{Investimentos pré-operacionais}

\begin{tabular}{| l | l |}
  \hline
  \textbf{Descrição} & \textbf{Custo (R\$)}\\ \hline
  Despesas de legalização & 500,00\\ \hline        
\end{tabular}
  
\section{Investimento total}

  \subsection{Resumo de investimentos}
  
  \begin{tabular}{| l | l | l |}
    \hline
    \textbf{Investimentos} & \textbf{Valor (R\$)} & \textbf{Fração do total} (\%)\\ \hline
    Fixos & 8750,00 & 84\\ \hline
    Capital de giro & 1142,00 & 11\\ \hline
    Pré-operacionais & 500 & 5\\ \hline
    \textbf{Total} & 10392,00 & 100\\ \hline
  \end{tabular}
  
  \subsection{Resumo de recursos}
  
  \begin{tabular}{| l | l | l |}
    \hline
    \textbf{Fonte de recursos} & \textbf{Valor (R\$)} & \textbf{Fração do total} (\%)\\ \hline
    Próprios & 0 & 0\\ \hline
    Terceiros & 10392,00 & 100\\ \hline
    Outros & 0 & 0\\ \hline
    \textbf{Total} & 10392,00 & 100\\ \hline
  \end{tabular}
