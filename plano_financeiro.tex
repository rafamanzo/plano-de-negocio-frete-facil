\chapter{Plano Financeiro}\label{financeiro}

\section{Estimativa dos investimentos fixos}

  \subsection{Máquinas e equipamentos}
  \begin{tabular}{| l | l | l | l |}
    \hline
    \textbf{Descrição} & \textbf{Quantidade} & \textbf{Valor Unitário (R\$)} & \textbf{Total (R\$)}\\ \hline
    Computador pessoal & 5 & 1500,00 & 7500,00\\ \hline
    \multicolumn{3}{| l |}{\textbf{Subtotal}} & 7500,00\\ \hline
  \end{tabular}
  
  \subsection{Móveis e utensílios}
  \begin{tabular}{| l | l | l | l |}
    \hline
    \textbf{Descrição} & \textbf{Quantidade} & \textbf{Valor Unitário (R\$)} & \textbf{Total (R\$)}\\ \hline
    Cadeira & 5 & 100,00 & 500,00\\ \hline
    Mesa & 5 & 150,00 & 750,00\\ \hline
    \multicolumn{3}{| l |}{\textbf{Subtotal}} & 1250,00\\ \hline
  \end{tabular}

  \subsection{Total}
  Então, totalizando um investimento inicial de R\$8750,00
  
\section{Capital de giro}

  \subsection{Infraestrutura compuitacional inicial}
  
  \begin{tabular}{| l | l | l | l |}
    \hline
    \textbf{Descrição} & \textbf{Quantidade} & \textbf{Valor Unitário (R\$)} & \textbf{Total (R\$)}\\ \hline
    Reserva de 1 ano - EC2 small & 3 & 334,00 & 1002,00\\ \hline
    Reserva de 1 ano - RDS small & 1 & 140,00 & 140,00\\ \hline
    \multicolumn{3}{| l |}{\textbf{Total}} & 1142,00\\ \hline
  \end{tabular}
  
  \subsection{Estimativas de entrada e saída}
  
    \subsubsection{Prazo médio de vendas}
    
    \begin{tabular}{| l | l | l | l |}
      \hline
      \textbf{Fonte} & \textbf{\% da receita} & \textbf{Dias para receber} & \textbf{Média ponderada de dias}\\ \hline
      Assinatura & 75 & 0 & 0\\ \hline
      Participação no frete & 25 & 30 & 7,5\\ \hline
      \multicolumn{3}{| l |}{\textbf{Prazo médio total (dias)}} & 7.5\\ \hline
    \end{tabular}
    
    \subsubsection{Prazo médio de compras}
    Como as compras basicamente serão de infraestrutura na nuvem, cuja forma de pagamento, em geral, é por cartão de crédito internacional, podemos considerar um prazo médio de pagamento de 15 dias até o fechamento da fatura.
    
  \subsection{Totalizando}
  Portanto, a empresa deve receber os recursos para pagar suas contas antes de, de fato, ter que pagá-las, assim sendo a empresa, desde que tudo corra bem, não tem necessidade um caixa mínimo para realizar suas atividades.
  
\section{Investimentos pré-operacionais}

\begin{tabular}{| l | l |}
  \hline
  \textbf{Descrição} & \textbf{Custo (R\$)}\\ \hline
  Despesas de legalização & 500,00\\ \hline        
\end{tabular}
  
\section{Investimento total}

  \subsection{Resumo de investimentos}
  
  \begin{tabular}{| l | l | l |}
    \hline
    \textbf{Investimentos} & \textbf{Valor (R\$)} & \textbf{Fração do total} (\%)\\ \hline
    Fixos & 8750,00 & 84\\ \hline
    Capital de giro & 1142,00 & 11\\ \hline
    Pré-operacionais & 500 & 5\\ \hline
    \textbf{Total} & 10392,00 & 100\\ \hline
  \end{tabular}
  
  \subsection{Resumo de recursos}
  
  \begin{tabular}{| l | l | l |}
    \hline
    \textbf{Fonte de recursos} & \textbf{Valor (R\$)} & \textbf{Fração do total} (\%)\\ \hline
    Próprios & 0 & 0\\ \hline
    Terceiros & 10392,00 & 100\\ \hline
    Outros & 0 & 0\\ \hline
    \textbf{Total} & 10392,00 & 100\\ \hline
  \end{tabular}
  
\section{Estimativa de faturamento mensal da empresa}\label{sec:faturamento}

A Frete Fácil faturará cobrando assinaturas mensais (R\$ 100,00) dos transportadores e uma porcentagem (10\%) sobre o valor de cada frete. Num período de 12 meses são esperados um total acumulado de 915 assinaturas (somando os assinantes de cada mês) e ter intermediado um total de R\$ 297.000,00 (cerca de 3.000 fretes), resultando em faturamentos de:

\begin{itemize}
  \item \textbf{Assinaturas:} R\$ 91.500,00;
  \item \textbf{Participação em fretes:} R\$ 29.700,00.
\end{itemize}

Totalizando então um faturamento anual de R\$ 121.200,00. E mensalmente R\$ 10.100,00 em média.

\section{Estimativa do custo unitário terceirizações}\label{sec:custounitario}

O único serviço terceirizado serão os servidores, fornecidos pela \textit{Amazon Web Services}, totalizando em um ano um custo de R\$ 9.887,00.

Ponderando pelas proporções da receita (75\% assinaturas e outros 25\% participação em fretes), o custo unitário deste serviço prestado em cada fonte de receita é de:

\begin{itemize}
  \item \textbf{Assinatura:} R\$ 8,10;
  \item \textbf{Participação em frete:} R\$ 0,82 (em média).
\end{itemize}

\section{Estimativa dos custos de comercialização}\label{sec:custosdecomercializacao}
A empresa adotará o SIMPLES como regime de tributação e portanto estará sujeita a este e IRPJ (0,5\%) e CSLL (1\%). Considerando uma receita bruta de R\$ 121.200,00 e lucro antes do Imposto de renda de R\$ 2.107,00, temos:
\newline \newline
\begin{tabular}{| l | l |}
  \hline
  \textbf{Imposto} & \textbf{Valor (R\$)}\\ \hline
  SIMPLES & 6.666,00 \\ \hline
  CSLL & 21,07 \\ \hline
  IRPJ & 10,54 \\ \hline
  \textbf{Total:} & 6697,61 \\ \hline
\end{tabular}

\section{Custo das mercadorias vendidas}\label{sec:cmv}

Ainda considerando o período dos 12 meses para a estimativa das quantidades e os custos unitários da seção \ref{sec:custounitario} como o custo unitário de matéria-prima das fontes de receita, temos a seguinte tabela:

\begin{tabular}{| l | l | l | l |}
  \hline
  \textbf{Serviço} & \textbf{Estimativa de vendas} & \textbf{Custo unitário (R\$)} & \textbf{CMV (R\$)} \\ \hline
  Assinatura & 915 & 8,10 & 7411,50 \\ \hline
  Participação & 3000 & 0,82 & 2460,00 \\ \hline
  \multicolumn{3}{| l |}{\textbf{Total}} & 9871,50 \\ \hline
\end{tabular}

\section{Estimativa dos custos com mão-de-obra}\label{sec:mao-de-obra}

  \subsection{Salários}\label{subsec:salarios}
  \begin{tabular}{| l | l | l | l |}
    \hline
    \textbf{Função} & \textbf{Quantidade} & \textbf{Salário mensal (R\$)} & \textbf{Subtotal (R\$)}\\ \hline
    Desenvolvedor & 4 & 1.500,00 & 6.000,00 \\ \hline
    Atendente & 1 & 1.000,00 & 1.000,00 \\ \hline
    \multicolumn{3}{| l |}{\textbf{Total}} & 7.000,00 \\ \hline
  \end{tabular}

  \subsection{Encargos sociais}\label{subsec:encargossociais}
  
  Como o trabalho de desenvolvimento será o trabalho dos sócios da empresa, não haverão encargos sociais sobre este. Assim, justificando a inexistência de encargos sociais para estes.
  \newline \newline
  \begin{tabular}{| l | l | l | l |}
    \hline
    \textbf{Função} & \textbf{\% de encargos sociais} & \textbf{Encargos sociais (R\$)} &\textbf{Subtotal (R\$)} \\ \hline
    Desenvolvedor & 0 & 0,00 & 0,00\\ \hline
    Atendente & 100 & 1.000,00 & 1000,00\\ \hline
    \multicolumn{3}{| l |}{\textbf{Total}} & 1.000,00 \\ \hline
  \end{tabular}

  \subsection{Total (\ref{subsec:salarios} + \ref{subsec:encargossociais})}
  
  Portanto, totalizando \ref{subsec:salarios} e \ref{subsec:encargossociais} o custo mensal com mão-de-obra será de R\% 8.000,00.
  
\section{Estimativa com custos com depreciação}

  \subsection{Estimativas de valor e vida útil}
  \begin{tabular}{| l | l | l |}
    \hline
    \textbf{Ativo fixo} & \textbf{Valor do bem (R\$)} & \textbf{Vida útil (anos)} \\ \hline
    Computador pessoal & 7.500,00 & 3\\ \hline
    Móveis & 1.250,00 & 10\\ \hline
  \end{tabular}

  \subsection{Custos mensal e anual}
  \begin{tabular}{| l | l | l |}
    \hline
    \textbf{Ativo fixo} & \textbf{Depreciação anual (R\$)}  & \textbf{Depreciação mensal (R\$)} \\ \hline
    Computador pessoal & 2.500,00 & 208,35\\ \hline
    Móveis & 125,00 & 10,42\\ \hline
    \textbf{Total:} & 2725,00 & 218,77 \\ \hline
  \end{tabular}
  
\section{Estimativa de custos fixos operacionais mensais}\label{sec:custosfixos}

\begin{tabular}{| l | l |}
  \hline
  \textbf{Descrição} & \textbf{Custo total mensal (R\$)} \\ \hline
  Aluguel & 400,00 \\ \hline
  Água & 50,00 \\ \hline
  Eletricidade & 100,00 \\ \hline
  Internet & 100,00 \\ \hline
  Telefone & 100,00 \\ \hline
  Honorários do contador & 500,00 \\ \hline
  Serviço de e-mail & 10,00 \\ \hline
  \textbf{Total} & 1.260,00 \\ \hline
\end{tabular}

\section{Demonstrativo de resultados}
  
\begin{tabular}{| l | l | l | l |}
  \hline
  \textbf{Seção} & \textbf{Descrição} & \textbf{Valor (R\$)} & \textbf{Proporção (\%)} \\ \hline
  \ref{sec:faturamento} & 1. Receita total com vendas & 121.200,00 & \\ \hline \hline
   & 2. Custos variáveis totais & 112.568,61 & 100 \\ \hline
  \ref{sec:cmv} & (-) Custo das mercadorias vendidas & 9.871,00 & \\ \hline
  \ref{sec:custosdecomercializacao} & (-) Impostos sobre vendas & 6.697,61 & \\ \hline
  \ref{sec:mao-de-obra} & (-) Custos com mão-de-obra & 96.000,00 (8.000 X 12) & \\ \hline \hline
   & 3. Margem de contribuição (1 - 2) & 8631.39 & \\ \hline \hline
  \ref{sec:custosfixos} & 4. (-) Custos fixos totais & 15.120,00 & \\ \hline \hline 
   & 5. Resultado Operacional (Lucro/Prejuízo) (3 – 4) & 6.489,00 & \\ \hline
\end{tabular}
