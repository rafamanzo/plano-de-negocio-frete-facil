\documentclass{beamer}
\usepackage[T1]{fontenc}
\usepackage[utf8]{inputenc}
\usepackage{lmodern}
\usepackage[brazil]{babel}
\usepackage{color}

\usetheme{JuanLesPins}

\title{Frete Fácil}
\author{Caio Teixeira da Quinta\\
        Giancarlo Rigo\\
        Rafael Reggiani Manzo\\
        Willen Goulart}

\begin{document}

\maketitle

\section{Introdução}
\begin{frame}
  \frametitle{Frete Fácil}
  \framesubtitle{Missão}
  
  Intermediar a relação entre todo indivíduo e empresa, que precise que algo seja entregue com baixo custo e dentro do prazo, e autônomos e empresas de frete na cidade de São Paulo.

\end{frame}

\section{Sumário Executivo}
\begin{frame}
  \frametitle{}
  \framesubtitle{}

  \begin{center}
    {\huge\textbf{Sumário Executivo}}
  \end{center}
\end{frame}

\begin{frame}
  \frametitle{Sócios}
  \framesubtitle{Caio}

\end{frame}

\begin{frame}
  \frametitle{Sócios}
  \framesubtitle{Giancarlo}

\end{frame}

\begin{frame}
  \frametitle{Sócios}
  \framesubtitle{Rafael}

\end{frame}

\begin{frame}
  \frametitle{Sócios}
  \framesubtitle{Willen}

\end{frame}

\section{Análise de mercado}

\begin{frame}
  \frametitle{}
  \framesubtitle{}

  \begin{center}
    {\huge\textbf{Análise de Mercado}}
  \end{center}
\end{frame}

\begin{frame}
  \frametitle{Mercado atual}
  \framesubtitle{}

\end{frame}

\section{Plano de marketing}

\begin{frame}
  \frametitle{}
  \framesubtitle{}

  \begin{center}
    {\huge\textbf{Plano de Marketing}}
  \end{center}
\end{frame}

\begin{frame}
  \frametitle{Produtos}
  \framesubtitle{}

\end{frame}

\begin{frame}
  \frametitle{Precificação}
  \framesubtitle{}

\end{frame}

\begin{frame}
  \frametitle{Estratégias promocionais}
  \framesubtitle{}

\end{frame}

\begin{frame}
  \frametitle{Estrutura de comercialização}
  \framesubtitle{}

\end{frame}

\section{Plano operacional}
\begin{frame}
  \frametitle{}
  \framesubtitle{}

  \begin{center}
    {\huge\textbf{Plano Operacional}}
  \end{center}
\end{frame}

\begin{frame}
  \frametitle{Capacidade produtiva}
  \framesubtitle{Novas funcionalidades e atendimento a clientes}
  
  \begin{itemize}
    \item 160 horas de trabalho semanais no desenvolvimento do produto;
    \item Capacidade para até 175 empresas por mês;
    \item E intermediar até 1575 fretes por mês;
  \end{itemize}
    
\end{frame}

\begin{frame}
  \frametitle{Pessoal}
  
  Será contratada uma pessoa para se dedicar a dar suporte aos clientes.

\end{frame}

\section{Plano Financeiro}
\begin{frame}
  \frametitle{}
  \framesubtitle{}

  \begin{center}
    {\huge\textbf{Plano Financeiro}}
  \end{center}
\end{frame}

\begin{frame}
  \frametitle{Demonstrativo de resultados}
  \framesubtitle{Expectativa para o primeiro ano}
  
  \begin{small}
    \begin{tabular}{| l | l |}
      \hline
      \textbf{Descrição} & \textbf{Valor (R\$)}\\ \hline
      1. Receita total com vendas & \textcolor{green}{130.350,00}\\ \hline \hline
      2. Custos variáveis totais & \textcolor{red}{113.605,68}\\ \hline
      (-) Custo das mercadorias vendidas & \textcolor{red}{9.871,00}\\ \hline
      (-) Impostos sobre vendas & \textcolor{red}{7.734,68} \\ \hline
      (-) Custos com mão-de-obra & \textcolor{red}{96.000,00} (8.000 X 12)\\ \hline \hline
      3. Margem de contribuição (1 - 2) & \textcolor{green}{16.764,32}\\ \hline \hline
      4. (-) Custos fixos totais & \textcolor{red}{15.120,00} (1.260 X 12)\\ \hline \hline 
      5. Resultado Operacional (Lucro/Prejuízo) (3 – 4) & \textcolor{green}{1.624,32}\\ \hline
    \end{tabular}
  \end{small}
\end{frame}

\begin{frame}
  \frametitle{Indicadores de viabilidade}
  \framesubtitle{Ponto de equilíbrio}

  \begin{tabular}{| l | l |}
    \hline
    \textbf{Receita total:} & R\$ 130.350,00 \\ \hline
    \textbf{Custo variável total:} & R\$ 113.605,68 \\ \hline
    \textbf{Custo fixo total:} & 15.120,00 \\ \hline
  \end{tabular}
  
  \begin{itemize}
    \item \textbf{Índice da margem de contribuição:} $\frac{130350 - 113605,68}{130350} \cong 0.13$
    \item \textbf{Ponto de equilíbrio:} $\frac{15120}{0.13} = 116307,69$
  \end{itemize}
  
  Portanto, a empresa estará cobrindo seus custos a partir do momento em que atingir uma receita bruta total de R\$ 116.307,69.
\end{frame}

\begin{frame}
  \frametitle{Indicadores de viabilidade}
  \framesubtitle{Lucratividade}
  
  \begin{tabular}{| l | l |}
    \hline
    \textbf{Receita total:} & R\$ 130.350,00 \\ \hline
    \textbf{Lucro líquido:} & R\$ 10.753,75 \\ \hline
  \end{tabular}
  
  .\newline \newline
  
  Resultando em uma lucratividade de $\frac{10753,75}{130350} \cong 8\%$.
\end{frame}

\begin{frame}
  \frametitle{Indicadores de viabilidade}
  \framesubtitle{Rentabilidade e prazo de retorno}

  \begin{tabular}{| l | l |}
    \hline
    \textbf{Lucro líquido:} & R\$ 10.753,75 \\ \hline
    \textbf{Investimento total:} & R\$ 10.392,00 \\ \hline
  \end{tabular}
  
  .\newline \newline
  
  \begin{itemize}
    \item Resultando em uma lucratividade de $\frac{10753,75}{10392}*100 \cong 103\%$ ao ano.
    \item Desta forma recuperando o investimento antes do primeiro ano.
  \end{itemize}
\end{frame}

\section{Construção de cenários}
\begin{frame}
  \frametitle{}
  \framesubtitle{}

  \begin{center}
    {\huge\textbf{Construção de Cenários}}
  \end{center}
\end{frame}

\section{Avaliação estratégica}
\begin{frame}
  \frametitle{}
  \framesubtitle{}

  \begin{center}
    {\huge\textbf{Avaliação Estratégica}}
  \end{center}
\end{frame}

\section{Avaliação do plano de negócio}
\begin{frame}
  \frametitle{}
  \framesubtitle{}

  \begin{center}
    {\huge\textbf{Avaliação do plano de negócio}}
  \end{center}
\end{frame}
\end{document}
