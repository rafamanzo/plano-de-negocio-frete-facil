\chapter{Sumário Executivo}\label{sumario}

\section{Resumo dos principais pontos}
  A Frete Fácil será uma empresa de intermediação de serviços de frete, que combinará a necessidade de um cliente por um frete em um período de tempo, local e por um valor máximo, com a vontade de ums transportador de aumentar seus rendimentos e reduzir a ociosidade de sua frota.
\newline

Os principais clientes da Frete Fácil serão pequenas e médias transportadoras, assim como trabalhadores autônomos e motoentregadores.\newline

A empresa estará sediada em Barueri-SP, valendo-se da alíquota de 2\% de Imposto sobre Serviços (ISS), inferior a outros municípios brasileiros.\newline

O montante inicial a ser investido é de R\$10.392,00, já o faturamento mensal médio será de R\$10.100,00.
O lucro esperado é de 8\% e o tempo para que o capital retorne é inferior a um ano.\newline

\section{Dados dos empreendedores}
\subsection{Caio Teixeira da Quinta}
  Caio Teixeira da Quinta é graduando em Bacharelado de Ciência da Computação no Instituto de Matemática e Estatística da Universidade de São Paulo (IME-USP) e possui experiência em desenvolvimento de software. Caio será responsável pelo desenvolvimento dos softwares da Frete Fácil.

\subsection{Giancarlo Rigo}
  Giancarlo Rigo é graduando em Bacharelado de Ciência da Computação no Instituto de Matemática e Estatística da Universidade de São Paulo (IME-USP), possui experiência em desenvolvimento de software e na área de vendas. Giancarlo será responsável pelo área administrativa e vendas da empresa.

\subsection{Rafael Reggiani Manzo}
Rafael Reggiani Manzo é graduando em Bacharelado de Ciência da Computação no Instituto de Matemática e Estatística da Universidade de São Paulo (IME-USP) e possui vasta experiência em desenvolvimento de software. Rafael será responsável pelo desenvolvimento dos softwares da Frete Fácil.

\subsection{Willen Goulart}
Willen Goulart é graduando em Bacharelado de Ciência da Computação no Instituto de Matemática e Estatística da Universidade de São Paulo (IME-USP), possui experiência em desenvolvimento de software e empreendedor no mesmo setor de atividade. Willen será responsável pelo desenvolvimento dos softwares da Frete Fácil e auxiliará na administração da empresa.
        
\section{Dados do empreendimento}

\begin{itemize}
  \item \textbf{Nome Fantasia:} Frete Fácil
  \item \textbf{Razão Social:} Frete Fácil Ltda.
\end{itemize}

\section{Missão da empresa}
Oferecer a intermediação de serviços de fretes, possibilitando a redução de custos e o melhor aproveitamento da frota.

\section{Setores de atividade}
A Frete Fácil atuará no setor da atividade econômica de Prestação de Serviços.

\section{Forma jurídica}
A Forma Jurídica da empresa será a Sociedade Limitada.

\section{Enquadramento tributário}
Devido ao porte da empresa, e de maneira a tornar a empresa competitiva e mais facilmente administrável, o enquadramento tributário escolhido é o Simples Nacional. 

\subsection{Âmbito federal}
No âmbito federal o enquadramento tributário é o Regime Simples.

\subsection{Âmbito estadual}
Como a Frete Fácil será uma empresa de pequeno porte com receita bruta inferior a R\$60.000,00 por mês, adotaremos o ICMS – Regime Simplificado.
  
\subsection{Âmbito municipal}
A prefeitura deveremos pagar o Imposto sobre Serviços (ISS).

\section{Capital Social}
\begin{tabular}{| l | l | l | l |}
  \hline
  \textbf{ } & \textbf{Nome do Socio} & \textbf{valor (R\$)} & \textbf{\% de participação} \\ \hline
  & Caio Teixeira da Quinta & 2.598,00  & 25 \\ \hline
  & Giancarlo Rigo & 2.598,00  & 25 \\ \hline
  & Rafael Reggiani Manzo & 2.598,00  & 25 \\ \hline
  & Willen Goulart & 2.598,00  & 25 \\ \hline
  \textbf{Total} & & 10392,00 & 100 \\ \hline
\end{tabular} 

\section{Fonte de recursos}
  Os recursos iniciais para a empresa serão obtidos através de empréstimos bancários.
